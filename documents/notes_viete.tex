\documentclass{article}
\usepackage[latin1]{inputenc}
\usepackage{cancel}
\usepackage{alltt}

%\definecolor{ttblue}{RGB}{48,48,170}
\newdimen\topcrop
\topcrop=10cm  %alternatively 8cm if the pdf inclusions are in letter format
\newdimen\topcropBezier
\topcropBezier=19cm %alternatively 16cm if the inclusions are in letter format

%\setbeamertemplate{footline}[frame number]
\title{A summary of a proof for Vi�te's formula for \(\pi\)}
\author{Yves Bertot}
\date{January 2026}
\begin{document}
%\begin{frame}
%\frametitle{Properties of cosine}
The point is to prove
\[\pi = 2 \cdot \frac 2 {\sqrt{2}} \cdots \frac 2 {\sqrt{2 + \cdots \sqrt{2}}} \cdots\]
Where the right-hand side is an infinite product.
\begin{itemize}
\item \(\cos (2 x) = \cos^2 x - \sin^2 x = 2\cos^2 x - 1\)
\item therefore \(\cos^2 x = 1 +\cos (2 x)\)
\item for \(x\) such that \(0 \leq x < \pi\),  \(\cos \frac{x}{2} > 0\)
\(\cos \frac{x}{2} = \sqrt{\frac{1 +\cos x}{2}} = \frac{\sqrt {2 + 2\cos x}}2\)
\item More specifically \(\cos \frac{\pi}4 = \sqrt{\frac 1 2}=
\frac{\sqrt{2}} 2 \)
\item Repeating the process we have \(\cos \frac{\pi}{2 ^ {n + 2}} = \frac{\sqrt{2 + \sqrt {2 + \cdots \sqrt{2}}}}{2}\) (\(n + 1\) occurrences of the square-root symbol)
\item \(\sin 2 x = 2 \sin x \cos x\)
\item Therefore, \(\sin \frac x 2 = \frac {\sin x}{2 \cos \frac x 2}\)
\item And \(\sin \frac x 4 = \frac {\sin \frac x 2}{2 \cos \frac x 4} =
\frac{\sin x}{2 \cos \frac x 2 \times 2 \cos \frac x 4}\)
\item Repeating the process \(\sin \frac x{2 ^ n} =
  \frac{\sin x} {\prod_{i=1}^{n} 2 {\cos \frac{x}{2 ^ i}}}\)
\item Specializing for \(x=\frac \pi 2\), we can use the fact that \(\sin \frac \pi 2 = 1\) to obtain \(\sin \frac \pi {2 ^ {n + 1}} = \frac 1 {\prod_{i = 1}^n {\cos \frac{\pi}{2 ^ {i + 1}}} \times 2 ^ n}\)
\item Multiplying both side by \(2 ^ {n + 1}\) we obtain \[2 ^ {n + 1} sin \frac{\pi}{2 ^ {n+1}} = \frac 2 {\prod_{i = 1}^n {cos \frac \pi {2 ^ {i + 1}}}} =
2 \cdot \frac 2 {\sqrt{2}} \cdots \frac 2 {\sqrt{2 + \cdots \sqrt{2}}}\]
The iterated product on the right-hand side is not infinite, it contains exactly \(n\) fractions with square roots at the denominator.  The last one contains \(n\) square root signs.  To compute it, one only needs to compute \(n\) square roots, because each denominator appears as a subformula of the denominator
that follows.
\item We can now turn our attention to the sequence \(n \mapsto 2 ^ {n + 1} \sin \frac \i {2 ^ {n + 1}}\).  Since \(\lim_{x \rightarrow 0} \frac{\sin x}{x} = 1\),
we have \(lim_{n \rightarrow \infty} \frac{\pi}{2 ^ {n}} = 0\), we can deduce
that \[\lim_{n \rightarrow \infty} 2 ^ {n + 1} \sin \frac{\pi}{2 ^ {n + 1}} = \pi\]
\item If we define the functions \(viete_1\) and \(viete\) for natural number
inputs, such that
\(viete_1 (0) = \sqrt 2\) and \(viete_1 (n) = \sqrt{2 + viete (n - 1)}\) when
\(n - 1\) is a natural number and
\(viete (n) = \frac{2}{viete_1(n)}\), then the cosine equality can be rewritten
as \(\cos \frac{\pi}{2 ^ {n + 2}} = \frac{viete_1(n)}{2}\) and the long product as \(2 \times \prod_{i = 0}^n viete(i)\)
\item Puttin all notations together, the infinite product can be rewritten as
\[\lim_{n \rightarrow \infty} 2 \times \prod_{i = 0}^n viete(i) = \pi\]
\end{itemize}


\end{document}


%%% Local Variables: 
%%% mode: latex
%%% TeX-master: t
%%% End: 
