\documentclass[compress]{beamer}
\usepackage[latin1]{inputenc}
\usepackage{cancel}
\usepackage{alltt}
\newdimen\topcrop
\topcrop=10cm  %alternatively 8cm if the pdf inclusions are in letter format
\newdimen\topcropBezier
\topcropBezier=19cm %alternatively 16cm if the inclusions are in letter format

\definecolor{craneblue}{rgb}{0.2,0.2,0.7}
\setbeamertemplate{footline}[frame number]
\title{A single number type for Math education\\in Type Theory}
\author{Yves Bertot}
\date{March 2025}
\mode<presentation>
\begin{document}

\maketitle
\begin{frame}
\frametitle{The context}
\begin{itemize}
\item Efforts to use theorem provers like Lean, Isabelle, or Rocq in teaching
\item language capabilities, documentation, error messages
\begin{itemize}
\item Strong inspiration: Waterproof
\item similar experiment on Lean (Lean Verbose, for instance)
\end{itemize}
\item Our contention: the contents needs adapting
\begin{itemize}
\item Several types for numbers, several versions of each operation
\item Coercions may be hidden, they can still block some operations
\end{itemize}
\item Typing helps young mathematicians, but not the type of natural numbers
\end{itemize}
\end{frame}
\begin{frame}
\frametitle{Characteristics of the natural numbers}
\begin{itemize}
\item Positive sides
\begin{itemize}
\item An inductive type
\item computation by reduction (faster than rewriting)
\item Proof by induction as instance of a general scheme
\item Recusive definitions are almost natural
\end{itemize}
\item Negative sides
\begin{itemize}
\item Subtraction is odd: the value of \(3-5\) is counterintuitive
\item The status of function/constructor {\tt S} is a hurdle for students
\item In Coq, {\tt S 4} and {\tt 5} and interchangeable,
   but\\ {\tt S x} and {\tt x + 1} are not
\item The time spent to learn pattern matching is not spent on math
\end{itemize}
\end{itemize}
\end{frame}
\begin{frame}
\frametitle{Numbers in the mind of math beginners}
\begin{itemize}
\item Starting at age 12, kids probably know about integer, rational, and
real numbers
\item \(3 - 5\) exists, you don�t need to think about the type of \(3\) and
  \(5\), and the value is not \(0\)
\item Computing \(127 - 42\) yields a natural number, \(3 - 5\) an integer,
and \(1 / 3\) a rational
\item \(42 / 6\) yields a natural number
\item These perception are {\em right}, respecting them is time efficient
\end{itemize}
\end{frame}
\begin{frame}
\frametitle{Proposal}
\begin{itemize}
\item Use only one type of numbers: real numbers
\begin{itemize}
\item Chosen to be intuitive for students at end of K12
\item Including the order relation
\end{itemize}
\item View other known types as subsets
\item Include stability laws in silent proof automation
\item Strong inspiration: the PVS approach
\begin{itemize}
\item However PVS is too aggressive on automation for education
\end{itemize}
\item Natural numbers, integers, still used in the background
\item Only postpone the introduction of {\tt nat} as a type
\end{itemize}
\end{frame}
\begin{frame}
\frametitle{Plan}
\begin{itemize}
\item Review of usages of natural numbers and integers
\item Defining subsets of \(\mathbb R\) for inductive types
\item Ad hoc proofs of membership
\item Recursive definitions and iterated functions
\item Finite sets and big operations
\item Minimal set of tactics
\item Practical examples, around Fibonacci, factorials and binomials
\end{itemize}
\end{frame}
\begin{frame}
\frametitle{Usages of natural numbers and integers}
\begin{itemize}
\item A basis for proofs by induction
\item Recursive sequence definition
\item iterating an operation a number of time \(f^n(k)\)
\item The sequence \(0\ldots n\)
\item indices for finite collections, \(a_1, \ldots a_n\)
\item indices for iterated operations \(\sum_{i=m}^{n} f(i)\)
\item Specific to Coq+Lean+Agda: constructor normal forms as targets of
reduction
\item In Coq real numbers, numbers 0, 1, ..., 37, ... rely on the inductive
type of integers for representation
\end{itemize}
\end{frame}
\begin{frame}
\frametitle{Defining subsets of \(\mathbb R\) for {\tt nat}, {\tt Z}}
\begin{itemize}
\item Use inductive predicates
\item Inherit the induction principle
\end{itemize}
\end{frame}
\begin{frame}[fragile]
\frametitle{Inductive predicate in Coq}
\begin{alltt}
Require Import Reals.
Open Scope R_scope.

Inductive Rnat : R -> Prop :=
  Rnat0 : Rnat 0
| Rnat_succ : forall n, Rnat n -> Rnat (n + 1).
\end{alltt}
Generated induction principle:
\begin{alltt}
nat_ind
     : forall P : R -> Prop,
       P 0 ->
       (forall n : R, Rnat n -> P n -> P (n + 1)) ->
       forall r : R, \textcolor{craneblue}{Rnat} r -> P r
\end{alltt}
\end{frame}
\begin{frame}[fragile]
\frametitle{Degraded typing}
\begin{itemize}
\item Stability laws provide automatable proofs of membership
\end{itemize}
\begin{small}
\begin{alltt}
Existing Class Rnat.

Lemma Rnat_add x y : Rnat x -> Rnat y -> Rnat (x + y).
Proof. ... Qed.

Lemma Rnat_mul x y : Rnat x -> Rnat y -> Rnat (x * y).
Proof. ... Qed.

Lemma Rnat_pos : Rnat (IZR (Z.pos _)).
Proof. ... Qed.

Existing Instances Rnat0 Rnat_succ Rnat_add Rnat_mul Rnat_pos.
\end{alltt}
\end{small}
\begin{itemize}
\item {\tt Rnat x -> Rnat ((x + 2) * x)} can be solved automatically
  (and silently).
\end{itemize}
\end{frame}
\begin{frame}
\frametitle{Ad hoc proofs of membership}
\begin{itemize}
\item When \(n, m \in {\mathbb N}\) and \(m \leq n\), \((n - m) \in {\mathbb N}\) can
also be proved
\item This requires an explicit proof
\item Probably good in a training context for students
\item division is similar
\end{itemize}
\end{frame}
\begin{frame}
\frametitle{Recursive functions}
\begin{itemize}
\item recursive sequences are a typical introductory subject
\item As an illustration, let us consider the {\em Fibonacci} sequence\\
{\em The Fibonacci sequence is the function \(F\) such that \(F_0= 0\), \(F_1=1\), and \(F_{n+2} = F_{n} + F_{n + 1}\) for every natural number \(n\)}
\item Proof by induction and the defining equations are enough to
{\em study} a sequence
\item But {\em def{}ining} is still needed
\item Solution: define a recursive definition command using Elpi
\end{itemize}
\end{frame}
\begin{frame}[fragile]
\frametitle{A recursive definition command for \(\mathbb R\) functions}
\begin{itemize}
\item The definition in the previous slide can be generated
\item Taking as input the equations
\item The results of the definition are in two parts
\begin{itemize}
\item The function of type {\tt R -> R}
\item The proof of the logical statement for that function
\end{itemize}
\end{itemize}
\begin{small}
\begin{verbatim}
Recursive (def fib such that
             fib 0 = 0 /\
             fib 1 = 1 /\
             forall n : R, Rnat (n - 2) ->
                fib n = fib (n - 2) + fib (n - 1)).
\end{verbatim}
\end{small}
\begin{itemize}
\item Need to distinguish the type and the domain of definition
\end{itemize}
\end{frame}
\begin{frame}
\frametitle{Compute with real numbers}
\begin{itemize}
\item {\tt Compute 42 - 67.} in plain Coq yields a puzzling answer
\begin{itemize}
\item Tons of {\tt R1}, {\tt +}, {\tt *}, and parentheses.
\item And to boot, the subtraction is left unexecuted
\end{itemize}
\end{itemize}
\end{frame}
\begin{frame}
\frametitle{Compute with real numbers: our solution}
\begin{itemize}
\item New command {\tt R\_compute}.
\item {\tt R\_compute (42 - 67).} succeeds and displays {\tt -25}
\item {\tt R\_compute (fib (42 - 67)).} fails!
\begin{itemize}
\item Respecting the fact that {\tt fib} is only defined for natural number inputs
\end{itemize}
\item {\tt R\_compute (fib 42) th\_name.} succeeds and saves a proof of equality.
\end{itemize}
\end{frame}
\begin{frame}
\frametitle{Big sums and products}
\begin{itemize}
\item Taking inspiration from Mathematical components
\item {\tt \char'134{}sum\_(a <= i < b) f(i)}
\begin{itemize}
\item Also {\tt \char'134{}prod}
\end{itemize}
\item Well-typed when {\tt a} and {\tt b} are real numbers\\
\item Relevant when \({\tt a} < {\tt b}\)
\item This needs a hosts of theorems
\begin{itemize}
\item Chipping off terms at both ends
\item Cutting in the middle
\item Shuffling the indices
\end{itemize}
\item Mathematical Components {\tt bigop} library provides a guideline
\item Future work: {\tt \char'134{}sum\_(a <= i <= b) f(i)} is more idiomatic
\end{itemize}
\end{frame}
\begin{frame}
\frametitle{Manipulating bound variables}
\[\sum_{i=0}^{10}\sqrt{i^2} = \sum_{i=0}^{10}i\]
\begin{itemize}
\item Is the sentence {\em replace \(\sqrt{i^2}\) with \(i\)} acceptable?
\item Can this sentence server as the start of the argument?
\item Does the proof language of the theorem prover allow mentioning \(i\)?
\item In the outer context, \(\sqrt{i ^ 2}\) is flawed, as \(i\) does not exist
\item In the mind of the teacher \(\sqrt{i ^ 2}\) exists, because it is written on the board!
\end{itemize}
\end{frame}
\begin{frame}
\frametitle{Examples of topics made easier}
\begin{itemize}
\item A study of factorials and binomial numbers
\begin{itemize}
\item Efficient computation of factorial numbers
\item Proofs relating the two points of view on binomial numbers,
   ratios or {\em Pascal's triangle} recursive definition
\item A proof of the expansion of \((x + y) ^ n\)
\end{itemize}
\item A study the fibonacci sequence
\begin{itemize}
\item \({\cal F}(i) = \frac{\phi ^ i - \psi ^ i}{\phi - \psi}\) (\(\phi\) golden ratio)
\end{itemize}
\end{itemize}
\end{frame}
\begin{frame}
\frametitle{The ``17'' exercise}
\begin{itemize}
\item Prove that there exists an \(n\) larger than 4 such that\\
\[\left(\begin{array}{c}n\\5\end{array}\right) =
  17 \left(\begin{array}{c}n\\4\end{array}\right)\]\\
(suggested by S. Boldo, F. Cl�ment, D. Hamelin, M. Mayero, P. Rousselin)
\item Easy when using the ratio of factorials and eliminating common
sub-expressions on both side of the equality\\
\[\frac{\cancel{n!}}{\cancel{(n - 5)!}\cancel{5!}_5} = 17 \frac{\cancel{n!}}{\cancel{(n-4)!}_{(n-4)}\cancel{4!}}\]
\item They use the type of natural numbers and equation\\
\[\left(\begin{array}{c}n\\p + 1\end{array}\right) \times (p + 1) =
  \left(\begin{array}{c}n\\p\end{array}\right) \times (n - p)\]\\
\end{itemize}
\end{frame}
\end{document}


%%% Local Variables: 
%%% mode: latex
%%% TeX-master: t
%%% End: 
