\documentclass[compress]{beamer}
\usepackage[latin1]{inputenc}
\usepackage{cancel}
\usepackage{alltt}
\newdimen\topcrop
\topcrop=10cm  %alternatively 8cm if the pdf inclusions are in letter format
\newdimen\topcropBezier
\topcropBezier=19cm %alternatively 16cm if the inclusions are in letter format

\setbeamertemplate{footline}[frame number]
\title{A single number type for Math education\\in Type Theory}
\author{Yves Bertot}
\date{June 2024}
\mode<presentation>
\begin{document}

\maketitle
\begin{frame}
\frametitle{The context}
\begin{itemize}
\item Efforts to use theorem provers like Lean, Isabelle, or Rocq in teaching
\item language capabilities, documentation, error message
\begin{itemize}
\item Strong inspiration: Waterproof
\item similar experiment on Lean (Lean Verbose, for instance)
\end{itemize}
\item Our contention: the contents also play a role
\begin{itemize}
\item Several types for numbers, several versions of each operation
\item Coercions may be hidden, they can still block some operations
\item Type theory forces developers to define functions where they should be
underfined
\end{itemize}
\item Typing helps young mathematicians, but not the type of natural numbers
\end{itemize}
\end{frame}
\begin{frame}
\frametitle{Characteristics of the natural numbers}
\begin{itemize}
\item Positive sides
\begin{itemize}
\item An inductive type
\item computation by reduction (faster than rewriting)
\item Proof by induction as instance of a general scheme
\item Recusive definitions are mostly natural
\end{itemize}
\item Negative sides
\begin{itemize}
\item Subtraction is odd: the value of \(3-5\) is counterintuitive
\item The status of function/constructor {\tt S} is a hurdle for students
\item In Coq, {\tt S 4} and {\tt 5} and interchangeable,
   but\\ {\tt S x} and {\tt x + 1} are not
\item The time spent to learn pattern matching is not spent on math
\item Too much cognitive load
\end{itemize}
\end{itemize}
\end{frame}
\begin{frame}
\frametitle{Numbers in the mind of math beginners}
\begin{itemize}
\item Starting at age 12, kids probably know about integer, rational, and
real numbers
\item \(3 - 5\) exists as a number, it is not 0
\item Computing \(127 - 42\) yields a natural number, \(3 - 5\) an integer,
and \(1 / 3\) a rational
\item \(42 / 6\) yields a natural number
\item These perception are {\em right}, respecting them is time efficient
\end{itemize}
\end{frame}
\begin{frame}
\frametitle{Proposal}
\begin{itemize}
\item Use only one type of numbers: real numbers
\begin{itemize}
\item Chosen to be intuitive for students at end of K12
\item Including the order relation
\end{itemize}
\item View other known types as subsets
\item Include stability laws in silent proof automation
\item Strong inspiration: the PVS approach
\begin{itemize}
\item However PVS is too aggressive on automation for education
\end{itemize}
\item Natural numbers, integers, etc, still silently present in the background
\end{itemize}
\end{frame}
\begin{frame}
\frametitle{Plan}
\begin{itemize}
\item Review of usages of natural numbers and integers
\item Defining subsets of \(\mathbb R\) for inductive types
\item From \(\mathbb Z\) and \(\mathbb N\) to \(\mathbb R\) and back
\item Ad hoc proofs of membership
\item Recursive definitions and iterated functions
\item Finite sets and big operations
\item Minimal set of tactics
\item Practical examples, around Fibonacci, factorials and binomials
\end{itemize}
\end{frame}
\begin{frame}
\frametitle{Usages of natural numbers and integers}
\begin{itemize}
\item A basis for proofs by induction
\item Recursive sequence definition
\item iterating an operation a number of time \(f^n(k)\)
\item The sequence \(0\ldots n\)
\item indices for finite collections,
\item indices for iterated operations \(\sum_{i=m}^{n} f(i)\)
\item Specific to Coq+Lean+Agda: constructor normal forms as targets of
reduction
\item In Coq real numbers, numbers 0, 1, ..., 37, ... rely on the inductive
type of integers for representation
\begin{itemize}
\item In Coq, you can define {\tt Zfact} as an efficient equivalent of factorial and compute 100!
\end{itemize}
\end{itemize}
\end{frame}
\begin{frame}
\frametitle{Defining subsets of \(\mathbb R\) for inductive types}
\begin{itemize}
\item Inductive predicate approach
\begin{itemize}
\item Inherit the induction principle
\item Prove the existence of a corresponding natural or integer
\end{itemize}
\item Existence approach
\begin{itemize}
\item Show the properties normally used as constructors
\item Transport the induction principle from the inductive type to the predicate
\item Hurdle: not possible to use the induction tactics if the type of data
is not inductive
\end{itemize}
\end{itemize}
\end{frame}
\begin{frame}[fragile]
\frametitle{Inductive predicate in Coq}
\begin{alltt}
Require Import Reals.
Open Scope R_scope.

Inductive Rnat : R -> Prop :=
  Rnat0 : Rnat 0
| Rnat_succ : forall n, Rnat n -> Rnat (n + 1).
\end{alltt}
Generated induction principle:
\begin{alltt}
nat_ind
     : forall P : R -> Prop,
       P 0 ->
       (forall n : R, Rnat n -> P n -> P (n + 1)) ->
       forall r : R, Rnat r -> P r
\end{alltt}
\end{frame}
\begin{frame}
\frametitle{from \(\mathbb N\) and \(\mathbb Z\) to \(\mathbb R\) and back}
\begin{itemize}
\item Reminder: the types \(\mathbb N\) ({\tt nat}) and \(\mathbb Z\) ({\tt Z}),
should not be exposed
\item Injections {\tt INR} and {\tt IZR} already exist
\item New functions {\tt IRN} and {\tt IRZ}
\item definable using Hilbert's choice operator
\begin{itemize}
\item Requires {\tt ClassicalEpsilon}
\item use the inverse image for {\tt INR} and {\tt IZR} when {\tt Rnat}
or {\tt Rint} holds
\end{itemize}
\end{itemize}
\end{frame}
\begin{frame}[fragile]
\frametitle{Degraded typing}
\begin{itemize}
\item Stability laws provide automatable proofs of membership
\end{itemize}
\begin{small}
\begin{alltt}
Existing Class Rnat.

Lemma Rnat_add x y : Rnat x -> Rnat y -> Rnat (x + y).
Proof. ... Qed.

Lemma Rnat_mul x y : Rnat x -> Rnat y -> Rnat (x * y).
Proof. ... Qed.

Lemma Rnat_pos : Rnat (IZR (Z.pos _)).
Proof. ... Qed.

Existing Instances Rnat0 Rnat_succ Rnat_add Rnat_mul Rnat_pos.
\end{alltt}
\end{small}
\begin{itemize}
\item {\tt typeclasses eauto} or {\tt exact \_.} will solve automatically
   {\tt Rnat x -> Rnat ((x + 2) * x)}.
\end{itemize}
\end{frame}
\begin{frame}
\frametitle{Ad hoc proofs of membership}
\begin{itemize}
\item When \(n, m \in {\mathbb N}\) and \(m \leq n\), \((n - m) \in {\mathbb N}\) can
also be proved
\item This requires an explicit proof
\item Probably good in a training context for students
\item Similar for division
\end{itemize}
\end{frame}
\begin{frame}
\frametitle{Recursive functions}
\begin{itemize}
\item recursive sequences are a typical introductory subject
\item As an illustration, let us consider the {\em Fibonacci} sequence\\
{\em The Fibonacci sequence is the function \(F\) such that \(F_0= 0\), \(F_1=1\), and \(F_{n+2} = F_{n} + F_{n + 1}\) for every natural number \(n\)}
\item Proof by induction and the defining equations are enough to
{\em study} a sequence
\item But {\em def{}ining} is still needed
\item Solution: define a recursive definition command using Elpi
\end{itemize}
\end{frame}
\begin{frame}[fragile]
\frametitle{Definition of recursive functions}
\begin{itemize}
\item We can use a {\em recursor}, mirror of the recursor on natural numbers
\item {\tt Rnat\_rec : ?A -> (R -> ?A -> ?A) -> R -> ?A}
\item Multi-step recursion can be implemented by using tuples of the
  right size
\end{itemize}
\begin{small}
\begin{alltt}
(* fib 0 = 0  fib 1 = 1              *)
(* fib n = fib (n - 1) + fib (n - 2) *)

Definition fibr := Rnat_rec [0; 1]
   (fun n l => [nth 1 l 0; nth 0 l 0 + nth 1 l 0]).
\end{alltt}
\end{small}
\end{frame}
\begin{frame}[fragile]
\frametitle{Meta-programming a recursive definition command}
\begin{itemize}
\item The definition in the previous slide can be generated
\item Taking as input the equations (in comments)
\item The results of the definition are in two parts
\begin{itemize}
\item The function of type {\tt R -> R}
\item The proof the logical statement for that function
\end{itemize}
\end{itemize}
\begin{small}
\begin{verbatim}
Recursive (def fib such that
             fib 0 = 0 /\
             fib 1 = 1 /\
             forall n : R, Rnat (n - 2) ->
                fib n = fib (n - 2) + fib (n - 1)).
\end{verbatim}
\end{small}
\end{frame}
\begin{frame}
\frametitle{Compute with real numbers}
\begin{itemize}
\item {\tt Compute 42 - 67.} yields a puzzling answer
\begin{itemize}
\item Tons of {\tt R1}, {\tt +}, {\tt *}, and parentheses.
\end{itemize}
\item {\tt Compute (42 - 67)\%Z.} yields the right answer
\begin{itemize}
\item Except it is in the wrong type
\end{itemize}
\end{itemize}
\end{frame}
\begin{frame}
\frametitle{Compute with real numbers: our solution}
\begin{itemize}
\item New command {\tt R\_compute}.
\item {\tt R\_compute (42 - 67).} succeeds and displays {\tt -25}
\item {\tt R\_compute (fib (42 - 67)).} fails!
\item {\tt R\_compute (fib 42) th\_name.} succeeds and saves a proof of equality.
\begin{itemize}
\item Respecting the fact that {\tt fib} is only defined for natural number inputs
\end{itemize}
\item Implemented by exploiting a correspondance between elementary operations
on {\tt R}, {\tt Z} (with proofs)
\item ELPI programming
\item Mirror a recursive definition in {\tt R} to a definition in {\tt Z}
\item Correcness theorem of the mirror process has a {\tt Rnat} on the input.
\end{itemize}
\end{frame}

\begin{frame}
\frametitle{Finite sets of indices}
\begin{itemize}
\item Usual mathematical idiom : \(1 \ldots n\), \(0 \ldots n\), \((v_i)_{i = 1 \ldots n}\)
\item Provide a {\tt Rseq : R -> R -> R}
\begin{itemize}
\item {\tt Rseq 0 3 = [0; 1; 2]}
\end{itemize}
\item Using the inductive type of lists here
\item This may require explaining structural recursive programming to students
\item At least {\tt map} and {\tt cat} (noted {\tt ++})
\end{itemize}
\end{frame}
\begin{frame}
\frametitle{Big sums and products}
\begin{itemize}
\item Taking inspiration from Mathematical components
\item {\tt \char'134{}sum\_(a <= i < b) f(i)}
\begin{itemize}
\item Also {\tt \char'134{}prod}
\end{itemize}
\item Well-typed when {\tt a} and {\tt b} are real numbers\\
\item Relevant when \({\tt a} < {\tt b}\)
\item This needs a hosts of theorems
\begin{itemize}
\item Chipping off terms at both ends
\item Cutting in the middle
\item Shuffling the indices
\end{itemize}
\item Mathematical Components {\tt bigop} library provides a guideline
\end{itemize}
\end{frame}
\begin{frame}[fragile]
\frametitle{Iterated functions}
\begin{itemize}
\item Mathematical idiom : \(f ^ n\), when \(f : A -> A\)
\item We provide {\tt Rnat\_iter} whose numeric argument is a real number
\item Only meaning full when the real number satisfies {\tt Rnat}
\item Useful to define many of the functions we are accustomed to see
\item Very few theorems are needed to explain its behavior
\begin{itemize}
\item \(f^{n+m}(a) = f^{n}(f^{m}(a))\quad f^1(a) = f(a)\quad f^0(a) = a\)
\end{itemize}
\end{itemize}
\end{frame}
\begin{frame}
\frametitle{Minimal set of tactics}
\begin{itemize}
\item {\tt replace}
\begin{itemize}
\item {\tt ring} and {\tt field} for justifications
\item No need to massage formulas step by step through rewriting
\end{itemize}
\item {\tt intros}, {\tt exists}, {\tt split}, {\tt destruct} to handle
  logical connectives (as usual)
\item {\tt rewrite} to handle the behavior of all defined functions
 (and recursors)
\item {\tt unfold} for functions defined by students
\begin{itemize}
\item But we should block unfolding of recursive functions
\end{itemize}
\item {\tt apply} and {\tt lra} to handle all side conditions related to bounds
\item {\tt typeclasses eauto} to prove membership in {\tt Rnat}
\begin{itemize}
\item Explicit handling for subtraction and division
\end{itemize}
\item Possibility to add ad-hoc computing facilities for user-defined
\begin{itemize}
\item Relying on mirror functions computing on inductive {\tt nat} or {\tt Z}
\end{itemize}
\end{itemize}
\end{frame}
\begin{frame}
\frametitle{Demonstration time}
\begin{itemize}
\item A study of factorials and binomial numbers
\begin{itemize}
\item Efficient computation of factorial numbers
\item Proofs relating the two points of view on binomial numbers,
   ratios or recursive definition
\item A proof of the expansion of \((x + y) ^ n\)
\end{itemize}
\item A study the fibonacci sequence
\begin{itemize}
\item \({\cal F}(i) = \frac{\phi ^ i - \psi ^ i}{\phi - \psi}\) (\(\phi\) golden ratio)
\end{itemize}
\end{itemize}
\end{frame}
\begin{frame}
\frametitle{The ``17'' exercise}
\begin{itemize}
\item Prove that there exists an \(n\) larger than 4 such that\\
\[\left(\begin{array}{c}n\\5\end{array}\right) =
  17 \left(\begin{array}{c}n\\4\end{array}\right)\]\\
(suggested by S. Boldo, F. Cl�ment, D. Hamelin, M. Mayero, P. Rousselin)
\item Easy when using the ratio of factorials and eliminating common
sub-expressions on both side of the equality\\
\[\frac{\cancel{n!}}{\cancel{(n - 5)!}\cancel{5!}_5} = 17 \frac{\cancel{n!}}{\cancel{(n-4)!}_{(n-4)}\cancel{4!}}\]
\item They use the type of natural numbers and equation\\
\[\left(\begin{array}{c}n\\p + 1\end{array}\right) \times (p + 1) =
  \left(\begin{array}{c}n\\p\end{array}\right) \times (n - p)\]\\
\end{itemize}
\end{frame}
\end{document}


%%% Local Variables: 
%%% mode: latex
%%% TeX-master: t
%%% End: 
