\documentclass[compress]{beamer}
\usepackage[latin1]{inputenc}
\usepackage{cancel}
\usepackage{alltt}

\definecolor{ttblue}{RGB}{48,48,170}
\newdimen\topcrop
\topcrop=10cm  %alternatively 8cm if the pdf inclusions are in letter format
\newdimen\topcropBezier
\topcropBezier=19cm %alternatively 16cm if the inclusions are in letter format

\setbeamertemplate{footline}[frame number]
\title{Rewriting under binders, comfortably}
\author{Yves Bertot}
\date{June 2024}
\mode<presentation>
\begin{document}

\maketitle
\begin{frame}
\frametitle{Plan}
\begin{itemize}
\item difficulty proving \[\sum_{i = 0}^{n} i = \sum_{i = 0}^{n} \sqrt{i^2}\]
\item Formal proofs have several steps, 
\item Math teacher proofs are very different,
\item A proposed solution.
\end{itemize}
\end{frame}
\begin{frame}
\frametitle{The context}
\begin{itemize}
\item Mathematical constructions like integrals and iterated sums have
{\em bound variables}
\item From the formal point of view, a bound variable does not really exist
\item Type theory promotes {\em Leibniz} equality as the main tool
to reason modulo equality
\begin{itemize}
\item especially for rewriting
\end{itemize}
\item But Leibniz equality requires objects that \underline{really exist}
\end{itemize}
\end{frame}
\begin{frame}
\frametitle{Discrepancy in idioms}
\[\sum_{i = 0}^{n} i = \sum_{i = 0}^{n} \sqrt{i ^ 2}\]
\begin{itemize}
\item The math teacher's proof (I believe)
\begin{itemize}
\item {\em Replace \(\sqrt{i^2}\) with \(i\) in the right-hand side sum.}
\item {\em Note that the sum ranges over positive values}
\end{itemize}
\item The formally verified proof
\begin{enumerate}
\item Establish \(\forall i, 0 \leq i \leq n \Rightarrow i = \sqrt{i ^ 2}\)
\item For this, fix \(i\) such that \(0 \leq i \leq n\),
\item Then \(i = \sqrt{i ^ 2}\) (by some proof),
\item then apply the extensionality lemma for sums:
\[\forall f g, (\forall i, 0 \leq i \leq n \Rightarrow f(i) = g(i)) \Rightarrow
\sum_{i = 0}^{n} f(i) = \sum_{i = 0}^{n} g(i)\]
\end{enumerate}
\end{itemize}
\end{frame}
\begin{frame}
\frametitle{The curse of \(\alpha\)-conversion}
\begin{itemize}
\item There is no doubt that, if \(i\) exists and is larger than 0,
\(i = \sqrt{i^2}\),
\item Leibniz says: if \(n = m\), you can replace \(n\) with \(m\) in
any formula
\begin{itemize}
\item But the numbers \(i\) and \(\sqrt{i}\) \underline{do not even exist}
in the formula
\[\sum_{i = 0}^{n} \sqrt{i^2}\]
\item Bound variable names do not count for logical reasoning
 \[\sum_{i = 0}^{n} {\sqrt{i^2}} = 
\sum_{j = 0}^{n} {\sqrt{j^2}}\]
\item \(\sqrt{i ^ 2}\) does not occur in the right-hand side formula!
\item So you cannot use Leibniz' principle directly
\end{itemize}
\end{itemize}
\end{frame}
\begin{frame}
\frametitle{A preliminary solution}
\begin{itemize}
\item Make the sentence {\em Replace \(\sqrt{i^2}\) with \(i\) in the sum.}
understandable by the proof system
\item Do not work modulo \(\alpha\)-conversion
\end{itemize}
\begin{enumerate}
\item Recognize that \(\sqrt{i^2}\) is not well-formed because we are missing
a variable with the name \(i\)
\item By scanning the formula, detect that \(i\) is bound in at least one place,
\item Search for instances of \(\sqrt{i^2}\) in the multiple places where
this may occur
\item Do this again if there are nested binding patterns
\item Every time one enters inside an operator with bound variables, apply
a suitable ``extensionality'' theorem
\end{enumerate}
\end{frame}
\begin{frame}
\frametitle{Example using the solution}

\begin{center}
\Huge DEMO
\end{center}
\end{frame}

\begin{frame}
\frametitle{A prototype implementation}
\begin{itemize}
\item Required an extension of the {\tt Elpi} meta-programming language
\item Authorize passing ``open terms'' as argument to tactics
\begin{itemize}
\item An open term is well typed in an extension of the context
\item Example, if \(i\) does not exist in the context\\
 \(\sqrt{i^2}\) is not well-typed,\\
 but \(\lambda i : {\mathbb R}, \sqrt{i^2}\) is well typed
\end{itemize}
\end{itemize}

\end{frame}

\end{document}


%%% Local Variables: 
%%% mode: latex
%%% TeX-master: t
%%% End: 
